\documentclass[12pt, a4paper]{article}

% paragraphs with larger spacing
\usepackage{parskip}

% code syntax highlighting
\usepackage{minted}
% set syntax highlighting theme
\usemintedstyle{monokai}

\title{Maximum Subarray Sum Problem}
\author{AmeerTaweel}
\date{September 15, 2022}

\begin{document}

\maketitle

\newpage

\tableofcontents

\newpage

\section{Description}

Given an array of $n$ numbers, calculate the maximum subarray sum, which is the
largest possible sum of a sequence of consecutive values in the array. There may
be negative numbers in the array. Zero-length subarrays are allowed, so the
maximum subarray sum is always at least zero.

\section{Input}

The first line of input contains an integer $n$, the number of elements in the
array.

Then $n$ lines follow, each containing a number.

\begin{listing}[!ht]
\begin{minted}[bgcolor=black, linenos]{cpp}
int n;
cin >> n;

vector<int> arr(n);
for (int i = 0; i < n; i++) cin >> arr[i];
\end{minted}
\caption{Read Input}
\label{listing:in}
\end{listing}

\section{Output}

One number, the maximum subarray sum.

\begin{listing}[!ht]
\begin{minted}[bgcolor=black, linenos]{cpp}
cout << max_sum;
\end{minted}
\caption{Write Output}
\label{listing:out}
\end{listing}

\newpage

\section{Solutions}

\subsection{Three Loops}

\subsubsection{Algorithm}

This algorithm coputes the sum for all sequences and then outputs the maximum.

\begin{listing}[!ht]
\begin{minted}[bgcolor=black, linenos]{cpp}
int max_sum = 0;

for (int start = 0; start < n; start++) {
    for (int end = start; end < n; end++) {
        int sum = 0;
        // Calculate sum for sequence [start, end]
        for (int i = start; i <= end; i++) sum += arr[i];
        if (sum > max_sum) max_sum = sum;
    }
}
\end{minted}
\caption{Three Loops Algorithm}
\label{listing:alg-three-loops}
\end{listing}

\subsubsection{Time Complexity}

The algorithm above has three nested loops that iterate through the output, so
its time complexity is $O(n^3)$.

\newpage

\subsection{Two Loops}

\subsubsection{Algorithm}

This algorithm improves on the previous one by calculating the sum as the end
pointer moves. Instead of doing the whole calculation at each step.

\begin{listing}[!ht]
\begin{minted}[bgcolor=black, linenos]{cpp}
int max_sum = 0;

for (int start = 0; start < n; start++) {
    int sum = 0;
    for (int end = start; end < n; end++) {
        sum += arr[end];
        if (sum > max_sum) max_sum = sum;
    }
}
\end{minted}
\caption{Two Loops Algorithm}
\label{listing:alg-two-loops}
\end{listing}

\subsubsection{Time Complexity}

The algorithm above has two nested loops that iterate through the output, so
its time complexity is $O(n^2)$.

\newpage

\subsection{One Loop - Kadane's Algorithm}

\subsubsection{Algorithm}

\begin{listing}[!ht]
\begin{minted}[bgcolor=black, linenos]{cpp}
int max_sum = 0;

int sum = 0;
for (int i = 0; i < n; i++) {
    if (sum < 0) sum = 0;
    sum += arr[i];
    if (sum > max_sum) max_sum = sum;
}
\end{minted}
\caption{Kadane's Algorithm}
\label{listing:alg-one-loop}
\end{listing}

\subsubsection{Time Complexity}

The algorithm above has only one loop that iterate through the output, so its
time complexity is $O(n)$.

\end{document}
